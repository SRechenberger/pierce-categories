\documentclass{article}

\usepackage{amsmath}
\usepackage{amssymb}
\usepackage{hyperref}
\usepackage{mathpartir}
\usepackage{palatino}
\usepackage{tikz}


\newcommand{\cpo}{\mathbf{CPO}}
\newcommand{\id}{\emph{id}}

\begin{document}
\begin{enumerate}
\item[3.4.7]
  The ``category'' $\cpo$ has $\omega$-complete partial orders as objects and $\omega$-continuous functions as arrows.
  We prove $\cpo$ is a category by showing the identity and associative laws hold.
  \begin{itemize}
  \item[\emph{assoc}]
    We show that for any two arrows $f : A \rightarrow B$ and $g : B \rightarrow C$ in $\cpo$, their composition $g \circ f$ is also in $\cpo$.
    First, $f$'s monotonicity implies that for all $p_1 \sqsubseteq_A p_2$, we have $f(p_1) \sqsubseteq_B f(p_2)$.
    Now using $g$'s monotonicity, we have that $g(f(p_1)) \sqsubseteq_C g(f(p_2))$, or rather, $(g circ f)(p_1) \sqsubseteq (g \circ f)(p_2)$.
    This shows that $g \circ f$ is monotone.
    
    To see that $g \circ f$ is continuous, we observe:
    \begin{align*}
      (g \circ f) \sqcup p_n &= g(f(\sqcup p_n))
    \\\langle\emph{f continuous}\rangle    &= g(\sqcup f(p_n))
    \\\langle\emph{g continuous}\rangle    &= \sqcup g(f(p_n))
    \\                       &= \sqcup (g \circ f)(p_n)
    \end{align*}

  \item[\emph{identity}]
    For any partial order $P \in \cpo$, define the function $\id_P : P \rightarrow P$ as $\id_P(p) = p$ for all elements $p$ of the chain $P$.
    Clearly $p_1 \sqsubseteq p_2$ implies $\id_P(p_1) \sqsubseteq \id_P(p_2)$.
    Also $\id_P \sqcup p_n = \sqcup p_n = \sqcup id_P(p_n)$, so $\id_P$ is continuous and therefore in $\cpo$.

    It follows by definition that $\id_P$ is a unit for composition.
  \end{itemize}

\item[3.4.9]
\item[3.4.11]
  If $f$ is an embedding, there must be an $f^R$ such that $f^R \circ f = \id_A$.
  This $f^R$ must be unique because $\id_A$ is unique.
  Likewise, if we have a projection $f^R$ then by definition there exists an embedding $f$, and uniqueness follows because $\id_A$ is unique.

\end{enumerate}
\end{document}
