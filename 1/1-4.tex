\documentclass{article}

\usepackage{amsmath}
\usepackage{amssymb}
\usepackage{hyperref}
\usepackage{mathpartir}
\usepackage{palatino}
\usepackage{tikz}


\begin{document}
\begin{enumerate}
\item[1.4.6.1]
  \textit{Terminal objects are unique up to isomorphism}
  \begin{proof}
    Assume $t1$ and $t2$ are terminal objects in a category $\mathcal{C}$.
    Since for every object $c\in\mathcal{C}$ there exists a unique arrow from $c$ to each terminal object, there exist arrows $f:t1\rightarrow t2$ and $g:t2 \rightarrow t1$.
    Thus $g \circ f = id_{t1}$ and $f \circ g = id_{t2}$, and $f$ and $g$ define our isomorphism.
  \end{proof}

  A nearly-identical argument proves initial objects are unique up to isomorphism.
  Just flip the arrows.

\item[1.4.6.2]
  \begin{description}
  \item[Set $\times$ Set:]
  \item[]
    \textit{Initial Object:} $(\{\} \times \{\})$

    \textit{Terminal Objects:} $\{ (\{A\} \times \{B\} )\ |\ A,B \in \textbf{Set} \}$
    The arrows in \textbf{Set $\times$ Set} are pairs of \textbf{Set} arrows.
    So the arrows leaving the initial object for other objects in the category are the pairs of arrows leaving the initial object in \textbf{Set}.
    Similarly for the terminal objects.

  \item[]
  \item[Set$^\rightarrow$:]
  \item[]
    \textit{Initial Object:} $\{ f~|~f : \{\} \rightarrow \{\} \}$

    There is exactly one way to map the empty set to any other domain or range.
  \item[]
    \textit{Terminal Objects:} $\{ f~|~f : \{A\} \rightarrow \{B\},~ A,B \in \textbf{Set}\}$

    Any domain maps uniquely to a singleton domain.
    Likewise for ranges.

  \item[poset $\mathcal{P}$:]
  \item[]
    \textit{Initial Objects:} The min objects
    $\{ p \ |\ \forall p^\prime\in \mathcal{P},\ p \le p^\prime \}$
  \item[]
    \textit{Terminal Objects:} The max objects 
    $\{ p \ |\ \forall p^\prime\in \mathcal{P},\ p^\prime \le p \}$
  \end{description}

\item[1.4.6.3]
  
  The initial and terminal objects in a single-object category (with only the identity arrow) are the same.

  Any discrete category with more than one element has no initial or terminal objects.
  More generally, any category with disconnected components has no initial or terminal objects.

  A simple functional language with types $\textsc{Unit}$ and $\textsc{Bool}$
  and arrows 
  $true: \textsc{Unit} \rightarrow \textsc{Bool}$ and
  $false: \textsc{Unit} \rightarrow \textsc{Bool}$
  can be formalized as a category with no initial objects.
  Initial objects must have one unique arrow to every other object in the 
  category, and neither $\textsc{Unit}$ nor $\textsc{Bool}$ satisfy
  this criteria.

  This category also has no terminal objects, for 
  there are no arrows from $\textsc{Bool}$ to $\textsc{Unit}$
  and two arrows from $\textsc{Unit}$ to $\textsc{Bool}$.
  
\end{enumerate}
\end{document}
