\documentclass{article}

\usepackage{amsmath}

\begin{document}

\begin{enumerate}
\item[2.1.4]
  We want to show that the definitions \emph{maplist}:
  \begin{align*}
    \emph{maplist}(f)([]) &= []
    \\ \emph{maplist}(f)([x]) &= [f x]
    \\ \emph{maplist}(f)(L * L') &= \emph{maplist}(f)(L) * \emph{maplist}(f)(L')
  \end{align*}
  and \emph{maplist'}
  \begin{align*}
    \emph{maplist'}(f)([]) &= []
    \\ \emph{maplist'}(f)([x] * L) &= [f x] * \emph{maplist'}(f)(L)
  \end{align*}
  are equivalent.
  This follows by the associativity of *.
  The second definition forces right associativity and the first allows any parenthesization.

\item[2.1.10.1]
  \begin{itemize}
  \item Example 2.1.5, the forgetful functor on monoids, satisfies the definition of a functor because it takes the identity homomorphism to the identity on the underlying set and the result of $F (f;g)$ for homomorphisms $f : (M,\cdot,e) \rightarrow (M',\cdot',e')$ and $g : (M',\cdot', e') \rightarrow (M'',\cdot'',e'')$ is $F(f); F(g)$ where $F(f) : M \rightarrow M'$ and $F(g) : M' \rightarrow M''$ are defined on the underlying sets. 
    Thus identities and composition are preserved.
  \item Example 2.1.6, the identity functor, trivially preserves identities and composition.
  \item Example 2.1.7, the right product functor, takes the identity arrow for an object $B$ to the product $(id_B, id_A)$, which is the identity arrow for the resulting pair, and maps compositions to the composition of product arrows $(f,id_A); (g,id_A)$, which is well-defined in the product category.
  \end{itemize}

\item[2.1.10.3] The functors between monoids considered as one-object categories are the monoid homomorphisms.
\end{enumerate}

\end{document}
